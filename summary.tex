\documentclass[a4paper,10pt] {article}
\usepackage[utf8]{inputenc}
\usepackage{hyperref}
\usepackage{comment}
\usepackage{graphicx}
\usepackage[italian]{babel}
\usepackage[font=small,labelfont=bf]{caption}
\title{Summary-SWE}
\author{carlo Sindico}
\begin{document}
\pagenumbering{arabic}



\begin{titlepage}

\newcommand{\HRule}{\rule{\linewidth}{0.5mm}} % Defines a new command for the


\center % Center everything on the page
 
%----------------------------------------------------------------------------------------
%	HEADING SECTIONS
%----------------------------------------------------------------------------------------

\textsc{\LARGE Universit\`a degli Studi di Padova}\\[1.5cm] % Name of your

\textsc{\Large Laurea in Informatica}\\[0.5cm] % Major heading such as course name\textsc{\large Corso di Ingegneria del Software}\\[0.5cm] % Minor heading such as


%----------------------------------------------------------------------------------------
%	TITLE SECTION
%----------------------------------------------------------------------------------------

\HRule \\[0.4cm]
{ \huge  Riassunto di fine corso}\\[0.3cm] % Title of your document
\HRule \\[1.5cm]
 
%----------------------------------------------------------------------------------------
%	AUTHOR SECTION
%----------------------------------------------------------------------------------------

\begin{minipage}{0.4\textwidth}
\begin{flushleft} \large
\emph{Studente:}\\
Carlo Sindico% Your name
\end{flushleft}
\end{minipage}
~
\begin{minipage}{0.4\textwidth}
\begin{flushright} \large
\emph{Matricola:}
\textsc{1069322} % Supervisor's Name
\end{flushright}
\end{minipage}\\[4cm]

% If you don't want a supervisor, uncomment the two lines below and remove the

%\Large \emph{Author:}\\
%John \textsc{Smith}\\[3cm] % Your name

%----------------------------------------------------------------------------------------
%	DATE SECTION
%----------------------------------------------------------------------------------------

{\large 18/02/2017}\\[3cm] % Date, change the \today to a set date if you want to


%----------------------------------------------------------------------------------------
%	LOGO SECTION
%----------------------------------------------------------------------------------------

%\includegraphics{Logo}\\[1cm] % Include a department/university logo - this will

 
%----------------------------------------------------------------------------------------

\vfill % Fill the rest of the page with whitespace

\end{titlepage}
\newpage
\pagenumbering{arabic}

\section{Lezione 1 Introduzione al corso}

\begin{itemize}
\item Il software, se utile, ha vita lunga. Il ruolo di un informatico è ``al
contorno della programmazione'', in modo da non produrre codice usa e getta; deve
\textbf{eseguire un processo di manutenzione del software e del codice.} Tutte le
attivit\'a hanno un costo e non possono essere buttate al vento.

\item \textbf{Engineering}, ingegnere, scienziato; l'ingegnere è colui che dai
principi scientifici ne trae una finalità concreta, pratica. La scienza deve
scoprire questi principi, prepararli. Crea qualcosa che sia mantenibile nel tempo
(piramidi, ponti, statue).
\item \textbf{Software}, è un terreno molto giovane, nato durante la seconda
guerra mondiale. Il SWE molto dopo, non è un ramo della computer science, ma
è \textbf{una disciplina ingegneristica che mette insieme elementi e
conoscenze.}
\item \textbf{In assenza di principi ingegneristici, aimè si fanno solo "dei
pasticci"}. Molte conoscenze vengono da campi che non sono l'informatica:
		\begin{itemize}
\item Informatica, si vuole che un buon ingegnere del software conosca tutte le
competenze informatiche;
		\item Matematica, scienza di base che aiuta a risolvere problemi;
\item Scienze gestionali ed economia, è un attività di gruppo, correlazionale,
capire come gestire risorse,
		tempo, denaro, cogestire;
		\item Ingegneria, è un pezzo di un sistema complesso che passa informazioni.
		\item Psicologia, il software è rilasciato con interfacce molto orientate alle
persone, bisogna intercettare le aspettative di chi usa il software.
		\end{itemize}

\item \textbf{Ciclo di vita di un software}, per regola il software nasce e ha
una vita che termina con il ritiro. Dal momento
in cui nasce esso passa in diversi stati, che lo fanno transire. Il software
spende la maggior parte del suo tempo
in uno stato che si chiama manutenzione, in cui si possono correggere degli errori
e cambiare lo stato. Noi
vorremmo una manutenzione che sia il meno invasiva possibile.
	\item Esistono svariate forme di manutenzione:
	\begin{itemize}
		\item \textbf{Correttiva}, per correggere difetti eventualmente rilevati
		\item \textbf{Adattiva}, per adattare il sistema alla variazione dei requisiti
		\item \textbf{Evolutiva}, per aggiungere funzionalità al sistema	

\item \textbf{NOTA: il software muore quando non c'è più nessuna manutenzione
da
nessun developer o da nessuna comunità - utenza.}
	\end{itemize}

\item \textbf{Efficienza}, quante risorse ho impiegato per fare ciò che ho
richiesto; misura il consumo e cresce al diminuire
del consumo. La massima efficienza è il consumo zero, quindi da sola non basta.
Le risorse che si consumano
sono persone, tempo, denaro, materiale.

\item \textbf{Efficacia}, è una misura della conformità, il raggiungere
l'obiettivo atteso. Si è efficaci se si raggiunge con
rapidità l'obiettivo; non misura le risorse. Bisogna cercare l'ottimo tra
efficienza ed efficacia, trovare il massimo equilibrio, massimizzare gli obiettivi
che sono tra l'efficienza e l'efficacia. Questi due termini sono in contrasto tra
loro.

\item\textbf{Progetto, assignment (incarico)}, sono progetti, un incarico
contrattualizzato fra parti
e non più negoziabile, mentre tutto il resto è negoziabile. Verrà
enfatizzato molto l'aspetto
contrattuale. Assignment + commitment (impegno inderogabile); il lavoro viene dati
da un assignement
e trasformato in commitment.

\item \textbf{Engagement}, essere impegnati formalmente, impegno dal quale non si
pu\'o fallire. A volte i progetti
falliscono clamorosamente, per un sacco di motivi che il swe dovrebbe cercare alla
radice. Il progetto
potrebbe essere obsoleto dalla nascita, incapacità di chi ha l'impegno, oppure
esaurimento di tempo
e/o finanziamenti.
In questo caso vogliamo imparare come si fa a non fallire, a soddisfare gli
obiettivi entro i tempi e i costi
noti a priori. Quante ore produttive mi servono e in quante ore di calendario
posso soddisfare determinati
obiettivi (a prescindere dalle persone). Per fare ciò devo applicare principi
ingegneristici. Per imbarcarsi in
un progetto devo sapere che ce la posso fare.

\item \textbf{Best practice}, il miglior modo di fare le cose, la selezione di
ciò che è meglio fare in una certa professione.

\item \textbf{Stakeholder}, è una persona che conta, un portatore di interesse,
che ha influenza sul prodotto o sul processo.

\item DEFINIZIONE DI INGEGNERIA DEL SOFTWARE:

\item \textbf{approccio sistematico, disciplinato e quantificabile allo sviluppo,
uso, manutenzione e ritiro del software.}
	\begin{itemize}
\item \textbf{Sistematico}, agisco secondo un sistema, affronto il medesimo
problema sempre nello stesso modo (nel modo
		giusto); senza adottare approcci improvvisati o creativi.

\item \textbf{Disciplinato}, ciascuno fa il suo e nessuno fallisce, perché il
software engineering è un'attività collaborativa. Seguire in modo
		rigoroso la disciplina è nostro dovere, e  produce affidabilità.

\item \textbf{Quantificabile}, esprimere una quantità; se si è sistematici e
disciplinati, si può calcolare a priori quanto tempo è necessario per
svolgere quella determinata cosa. Si può essere quanti cabili solo se si è
sistematici e disciplinati.
	\end{itemize}

\item Si cerca un approccio con queste caratteristiche (4P):

\item  \textbf{People}, insieme delle persone che commissionano e ricevono;
\item \textbf{Product}, il software è parte di un sistema complesso;
\item \textbf{Process}, l'insieme di attività che svolgiamo ad un particolare
stato;
\item \textbf{Project}, attività specifiche svolte a fronte di un assignment
che diventa un commitment.

\item Tre strati:

\item \textbf{Customer}, il cliente, quello che ha un bisogno espresso o
inespresso;
\item \textbf{Solution}, dove esiste il customer c'è qualcuno che trova una
soluzione, fatta di requisiti (di cosa c'è
bisogno);
\item \textbf{Endeavor}, l'impegno per realizzare il software, il team interagisce
con gli stakeholder; il software engineering \'e un'attivit\'a
strettamente collaborativa, ho bisogno di un metodo di lavoro per il team. Way of
working.

SWE != PROGRAMMING, la programmazione \'e solo un elemento, e anche il meno
importante. Il
programmatore deve fare solo quello che viene chiesto (da un membro del team
stesso). Il programmatore
deve obbedire, non pu\'o essere creativo.


\item Principi etici:
\item \textbf{Considerare la qualit\'a come primo obiettivo}
\item \textbf{Produrre sw di qualit\'a \'e possibile}
\item \textbf{Aiutare il cliente a comprendere i suoi veri bisogni}
\item \textbf{Adottare i processi pi\'u adatti al progetto}
\item \textbf{Ridurre la distanza intellettuale tra il software e il problema da
risolvere}
\item \textbf{Essere proattivi nel cercare e rimuovere gli errori}
\item \textbf{Motivare, formare, far crescere le persone}

\end {itemize}


\section{Lezione 2 Processi Software}
\begin{itemize}
\item La tecnologia evolve ad una velocit\'a incontrollabile diventa obsoleta
rapidamente. \'e necessario capire la differenza tra le cose essenziali e le cose
aaccidentali. Tra le cose essenziali vi \'e la disciplina, mentre tra le cose
accidentali vi  \'e la tecnica e gli strumenti. \'e importante concentrarsi sugli
aspetti fondamentali e non accidentali. Occorre avere capacit\'a di impegno
concettuale, astrazione, di analisi e rigore (best practice). Vogliamo utilizzare
la traccia di chi ha fatto le cose in passato e le a fatte bene. Nessuno scopo
pu\'o essere raggiunto senza impegno.


\item \textbf{Progetto didattico}, ci sono due approcci ed uno spazio intermedio.
Come si pu\'o notare dal grafico, le \textbf{skills} (competenze) servono per
affrontare \textbf{challenges} sempre pi\'u alte.

\item \textbf{Cosa non \'e un progetto}, aggregare informazioni e attivit\'a e
processi per ottenere un prodotto che sembra funzionale, ma non \'e il "basta
che respiri", bisogna sostituire il principio di "by correction", con il principio
di "by construction" (costruire sapendo che funzioner\'a). Niente correzioni
in fase di sviluppo.

\item \textbf{Cosa \'e un progetto}:
	\begin{itemize}
	\item \underline{Pianificazione}, pianfica chi sa cosa vuole fare, \'e 
	l'essenza
	per controllarsi, per sapere se stiamo
		convergendo o divergendo. Ogni attivit\'a inizia con la pianifcazione
	\item \underline{Analisi dei requisiti}: si analizza ci\'o di cui si ha 
	bisogno.
	L'analisi ha un'importanza decisiva, bisogna
				capire il problema
				\item \underline{Progettazione}, si decide la forma della 
				soluzione
	\item \underline{Realizzazione}, dove sta anche la programmazione, che deve
	aderire al 100\% alla progettazione. Nella
	realizzazione attuo, ma non sono ancora sicuro che il risultato soddisfi il
	cliente
				\item \underline{Verifica e validazione}
	\item \underline{Manutenzione}, per la maggior parte della sua vita il 
	prodotto
	resta in manutenzione. Non esiste vita
	operativa senza manutenzione, non esiste software perfetto quindi esso deve 
	essere
	mantenuto.
	\item \underline{Qualit\'a}, uno dei principi su cui punteremo, perch\'e la 
	qualit\'a \'e
	possibile. Vorremmo quantificare la	qualit\'a in modo oggettivo.
	\end{itemize}
	
	
\item \textbf{Proccessi software}, attivit\'a coordinate, processi di ciclo di vita per far evolvere il software da uno stato all'altro.

Il software \'e una macchina a stati:
	\item Concezione
	\item Sviluppo
	\item Utilizzo
	\item Ritiro
	
Nella fase di ritiro il software cessa di esistere nel senso che non c'\'e pi\'u alcun tipo di supporto per quel prodotto.
Le transizioni sono strettamente e formalmente regolate.


\item \textbf{Modelli di ciclo di vita}

\item \textbf{Iterazione ed incremento}, iterazione significa che faccio la 
stessa operazione pi\'u volte , incremento significa aggiungere, \'e additivo. 
Sono incrementale solo se aggiungo, \'e un obiettivo molto importante da 
raggiungere (e potenzialmente distruttivo, potrebbe far perdere tempo). Iterare 
significa riprovare, non ripetere. Non si pu\'o togliere quando si \'e 
incrementali.

\item \textbf{Prototipo}, serve per imparare, \'e tipicamente usa e getta. Viene diviso in due categorie: quelli rivolti al cliente, per fargli capire un\' esempio di risultato finale, e quelli rivolti verso di noi per aiutarci a trovare una soluzione. Quelli rivolti verso di noi sono un costo, mentre quelli rivolti verso il cliente sono un valore aggiunto. Il primo impatto tra l'utente e il prodotto \'e l'interfaccia. I prototipi possono essere usa e getta" ma costano tempo.

\item \textbf{Riuso}, il software che gi\'a esiste \'e la maggior parte del nostro prodotto. L'informatico deve insegnare a riusare in modo sistematico e non opportunistico(nessun copia incolla). Il riuso \'e cosa saggia se so approvvigionarmi da un fornitore intelligente.

\item La \textbf{manutenzione} richiede gestione della storia, versionamento. 
Bisogna avere una storia del proprio prodotto. Bisogna spiegare e documentare 
la scelta ed avere una tecnica che salvi la storia (repository).

\item \textbf{L'efficienza} si vede dove vedo il consumo di risorse. 
L'efficacia si 
misura guardando i prodotti e vedendo se sono buoni o cattivi rispetto alla 
produzione. Un processo \'e un insieme di attivit\'a coordinate e coese (tutti 
hanno bisogno di tutti).

Standard di riferimento: ISO/IEC 12207.	
\end{itemize}
\section{Lezione 3 Processi Software}
\begin{itemize}
	\item Il rischio \'e tutto ci\'o che mina le nostre certezze, ci\'o che non 
	riusciamo a governare. Bisogna convivere con
	il rischio e mitigarlo, cercare di prevenire il danno.
	
	\textbf{Risk mitigation}, ma prima bisogna capire quali sono i rischi.
	Bisogna affrontarli in modo che siano "addomesticati".
	
	
	\item Un progetto non inizia se non con un'attivit\'a importante che faccia 
	emergere i requisiti, la maggior parte dei quali sono inespressi. (Gli 
	stakeholder non hanno la capacit\'a o la pazienza 
	di scrivere tutto ci\'o di cui hanno
	bisogno). Un fornitore di un prodotto deve intuire queste necessit\'a. 
	Questo si fa con analisi ed emersioni predefinite.
	
	\item \textbf{Verifica e validazione}, insieme li chiamiamo 
	\textbf{qualifica} ma sono due cose distinte ed occupano un sacco di
	tempo. (Consapevolezza di consegnare al customer un prodotto valido. Non 
	dobbiamo far fare a lui il debug).
	
	\'e irragionevole sviluppare software da zero (from scrutch).\'e necessario 
	puntare sul riuso, e non sul coding la quale \'e un'attivit\'a frazionaria 
	con un notevole rischio, quindi da evitare.
	
	\item \textbf{Un processo} \'e un insieme di attivit\'a complesse. Le 
	attivit\'a sono suddividibili in parti pi\'u piccole, chiamate
	task. \'e importante concentrarsi sulla Pianificazione, suddividendo il 
	tempo.
	 Quanto pi\'u piccolo \'e il compito da svolgere tanto pi\'u
	piccolo \'e il rischio. Interesse decisivo \'e spezzare le attivit\'a ad 
	una grana fine per ridurre i tempi, costi rischi. E' importante capire come 
	frantumare le attivit\'a.  
	
	
	\item  Un compito lo si assume avendo acquisito consapevolezza degli 
	strumenti e delle tecniche per portarlo
	a termine. Bisogna scegliere uno strumento che sia collaborativo.
	
	\item Un software non \'e fine a se stesso ma \'e parte di un sistema. Il 
	sistema lo compongono tutti coloro che sono chiamati a usarlo. 
	
	\item \textbf{Organizzazione} \'e un aggregato di persone che 
	agiscono secondo regole, sono sistematici, disciplinati e
	quantificabili. 
	
	\item Un'impresa certamente funziona a "pipeline". L'organizzazione \'e 
	divisa in settori, ciascuno con un proprio compito, che sono finalizzati a 
	un 
"flusso". Ogni settore risponde alle esigenze di attivit\'a 
	trasversali. \textbf{Le regole sono i processi}. 
	
	\item La conoscenza su come il software funziona deve essere scritta. I 
	processi vanno documentati.( Norme di progetto, che dicono come un ruolo va 
	svolto).

	\item I processi, che sono l'organizzazione, hanno bisogno di manutenzione, 
	non possono durare per sempre.
	 
	\item I processi produttivi devono avere un ciclo interno atto a 
	migliorarli 
	costantemente. Il ciclo interno di miglioramento \'e indicato con 
	l'acronimo 
	PCDA (o principio di Deming). 
	Questo ciclo \'e fatto di 4 attivit\'a a ciclo:
	\begin{itemize}
		\item \textbf{Plan}, non \'e il piano delle attivit\'a di processo, ma 
		il
		piano di miglioramento; pianifico ci\'o che produce
		efficienza ed efficacia;
		
		\item \textbf{Do}, stretta attuazione di ci\'o che ho pianificato. Devo 
		sapere rispetto a cosa migliorare;
		
		\item \textbf{Check}, guardo l'esito migliorativo ottenute dalle modi
		che fatte rispetto al piano;
		
		\item \textbf{Act}, ottenuto il risultato porto questo esito a 
		migliorare il processo. 
		Porto miglioramenti o, se fallisco, ragiono su come portare 
		miglioramenti.
		
		
		\item \textbf{Il PDCA serve a migliorare l'efficienza e l'efficacia 
		attraverso degli obiettivi di miglioramento.}
		
		Un ciclo di vita sono gli stati di maturit\'a che ha un software dalla 
		sua nascita al suo ritiro.
		
	\end{itemize}

\section{Lezione 4 Ciclo di vita del software}
\begin{itemize}
	\item \textbf{Coclio di vita di un software}, conoscere il ciclo di vita di 
	un software ci aiuta a capire la software engineering. 
	
	
	
	\item Il software va inteso in un senso molto pi\'u ampio di "programma", 
	serve per soddisfare i requisiti dei clienti. il software evolve
	nel tempo e raggiunge stati tramite transizioni scatenate da attivit\'a che 
	hanno il fine di far avanzare il software stesso. Si divide in 4 fasi:
	\begin{itemize}
	\item \textbf{Concezione}, quando qualcuno pensa che ci sia (o abbia) 
	bisogno di qualcosa
	\item \textbf{Sviluppo}
	\item \textbf{Utilizzo}
	\item \textbf{Ritiro}
	
	Questi stati cambiano attraverso attivit\'a specifiche, derivanti da 
	processi di cicli di vita.
	
	\end{itemize}

	\item \textbf{Fase}, \'e un periodo di tempo contiguo con un inizio ed una 
	fine. Un prodotto software si spalma sul tempo.
	
	Possiamo Immaginare il ciclo di vita su un asse temporale, dove le fasi 
	sono segmenti tra uno stato e l'altro.
	
	\item \textbf{documentazione}. Serve a produrre documenti che evidenziano 
	lo stato di maturit\'a di un prodotto software. Questo processo va 
	utilizzato sempre (tranne che nel ritiro).
	
	\item E' importante avere un ciclo di vita chiaro in testa, perch\'e averlo 
	ci fa capire quello che dobbiamo fare e se siamo in grado di farlo. 
	
	\item Il compito fondamentale \'e utilizzare un approccio 
	quantificabile, in modo da poter misurare gli obiettivi e le aspettative. 
	Cosa devo fare? 
	Perch\'e? 
	Sono capace in questo modo di puntare a una qualit\'a misurabile.
	
	
	\item \textbf{Conformit\'a}, un grado di corrispondenza ad aspettative 
	(efficacia). Misurare quanto soddisfiamo (siamo conformi) alle aspettative.
	
	\item \textbf{Maturit\'a}, \'e il grado di stabilit\'a con il quale siamo 
	in grado di svolgere le attivit\'a. Siamo maturi quando abbiamo acquisito 
	il "modus operandi".Ossia essere sistematici attraverso un processo di 
	maturit\'a.
	
	\item \textbf{Modello di ciclo}, astrazione del modo nel quale vediamo gli 
	stati e il loro avanzamento.
	
	\item \textbf{Non modello},ad esempio il "code and x", raggiungere la 
	correttezza "by correction", nel quale le azioni sono eseguite senza 
	organizzazione preordinata."Non so perch\'e ce l'ho fatta".
	
	\item \textbf{I modelli devono essere organizzati}. Si identificano 6 
	modelli organizzativi:
	\begin{itemize}
	\item  \textbf{Modello Waterfall, "a cascata"}. Non \'e ammesso un ritorno 
	ad una fase gi\'a visitata cos\'i come non \'e possibile risalire una 
	cascata. Fatto una volta il codice deve essere corretto. 	
	\textbf{Document driven}, documenti che spiegano la fiducia nelle 
	scelte. Prima la documentazione poi il software. Senza i documenti
	non si va avanti, \'e una pre-condizione. La post-condizione \'e avere un 
	documento approvato. Le fasi sono distinte e non si sovrappongono, tutto 
	\'e allineato in una linea temporale. Ancora oggi \'e un modello
	molto utilizzato. La frustrazione pi\'u grande \'e che il codice arriva 
	tardi. Si programma molto dopo e rispetto alla nostra abitudine \'e 
	frustrante. Non si realizzano  prototipi, bisogna immaginarseli.
	Questo si fa solo se tutti gli stakeholder sono d'accordo sulle pre e le 
	post.Questo modello funziona in un mercato non troppo competitivo, 
	tipicamente negli appalti pubblici.


	\item \textbf{Modello "incrementale"}, un modello di tipo incrementale 
	prevede ritorni. E' difficile in questo modello definire una fase.
	Fisso un quadro generale, inizio a sviluppare la base, poi torno in ciclo e 
	aggiungo delle cose. Ogni realizzazione aggiunge un pezzo. Esempio: 
	sviluppo separato di interfaccia grafica utente e base di dati. Procedendo 
	a cicli di incremento so esattamente quanti cicli far\'o. 
	
	
	\item \textbf{Modello "Iterativo"},un modello iterativo farebbe un passo 
	pi\'u indietro. Se non ritorno in analisi il problema \'e completamente 
	definito e non corro il rischio di buttare via cose. Nel modello 
	iterativo torno indietro da realizzazione ad analisi (fasi). Si riprova. In 
	questo modello non so dire con certezza quante iterazioni far\'o.
	Il modello iterativo lo uso quando gli stakeholder non sanno esattamente 
	cosa vogliono. Si va per tentativi. E' un modello molto rischioso;
	
	
	\item \textbf{Modello "Evolutivo"},nel modello evolutivo si ragiona su 
	un'analisi preliminare (schizzo,un'idea non precisa) poi si inizia
	a fare una versione. Insegue il futuro non ritirando il passato, ho tante 
	attivit\'a concorrenti. Se le cose
	che voglio fare non le so all'inizio una nuova versione "asfalta" molto del 
	passato. Si fa su un prodotto che ha la capacit\'a di assorbire molte 
	versioni sul mercato (esempio i browser). Questo modello lo attua chi pu\'o 
	sostenere molte versioni in parallelo (e quindi ha buone 
	capacit\'a finanziarie);
	
	
	\item \textbf{Modello "spirale"}, il problema \'e il rischio non studiato. 
	Bisogna usare un modello fortemente consapevole dei rischi. A partire 
	dall'inizio ci sono molti cicli ripartiti per capire meglio i rischi e solo
	dopo la spirale si apre. Per riuscire a trovare una via di uscita si 
	utilizzano i prototipi. Servono a capire se il rischio si pu\'o risolvere, 
	confermandoo se ci\'o che si \'e fatto \'e ragionevole. Questo modello 
	aspira a togliere i rischi; ha avuto molto uso in ambienti risk driven, in 
	progetti di carattere "sistema";
	
	
	\item \textbf{Modello "componenti"}, nasce sull'osservazione che molto di 
	quello che ci serve esiste gi\'a. Pensare che rifare da capo sia molto 
	probabilmente fallimentare. Ragionare per riuso. Componenti riusabili.
	La progettazione confronta il problema con le realt\'a esistenti. Adattare 
	i requisiti alle disponibilit\'a. 
	Si negozia con lo stakeholder. Adatto i requisiti al panorama delle 
	possibilit\'a e poi progetto riusando le possibilit\'a. La creativit\'a sta 
	nell'integrare e nell'usare bene ci\'o che esiste gi\'a. Richiede per\'o di 
	studiare il dominio.
	
	
	\item \textbf{Metodi agili}, opposizione al modello sequenziale. 4 principi 
	su cui si basa:
	\begin{itemize}
	\item Mettere in primo piano persone e iterazioni piuttosto che processi e 
	strumenti. Gli individui sono importanti ed \'e importante il modo in cui 
	interagiscono tra loro
	
	\item "Dei documenti me ne frego, basta che funzioni". La documentazione 
	non sempre corrisponde a software funzionante. E' importante che funzioni. 
	Ma nel lungo periodo far\'a soffrire chi dovr\'a prendere in mano il 
	software
	
	\item Avere un buon rapporto con il customer, coinvolgerlo, non ingessare 
	il rapporto
	
	\item Reattivit\'a piuttosto che pianificazione. Capacit\'a di adattamento 
	a cambiamenti delle situazioni.
	Il modello agile \'e maturato in qualcosa di plausibile, in alcune tecniche 
	che funzionano.
	
	
	Modello agile educato, tutto ruota intorno al "document user story", e 
	l'espressione precisa di quello
	che l'utente vuole in quel momento. Associato ad ogni user story produrre 
	una strategia che sia in grado di soddisfare quei bisogni. User story 
	associato a un insieme di cose da fare che dimostrino al customer
	che c\'e una soluzione (product backlog). Quello che si ottiene va verso il 
	customer o in feedback. C\'e un supervisore che decide l'urgenza strategica 
	(sprint). Tutti fanno un pezzo e poi lo consegnano. Ogni iterazione \'e 
	rapida e va molto verso le aspettative del cliente.
	Scrum daily, mucchio giornaliero, un contatto giornaliero dell'avanzamento. 
	Tecnica ragionevole per chi ha poca esperienza.
	\end{itemize}
\end{itemize}	

\end{itemize}
\section{Lezione 5 Gestione di progetto}
\begin{itemize}
	\item I requisiti sono l'elemento fondamentale nella maturazione di un 
	progetto; spesso non sono ben espressi o capiti dal cliente o da noi. 
	
	\item Vi sono vari passi (stadi di maturit\'a):
	\begin{itemize}
		
	\item \textbf{Conceived}, i requisiti vanno concepiti, nascono nel luogo 
	dove stanno gli stakeholder, ovvero tutti coloro che danno forma ad un 
	prodotto. Non significa "completamente formato", i requisiti vanno fatti
	emergere;
	
	\item \textbf{Bounded}, recintati, evitare che i requisiti si "espandano 
	come il gas", cercare di "inserire i requisiti
	dentro un recinto";
	
	\item \textbf{Coherent}, devo guardare i requisiti e devono appartenere 
	tutti allo stesso ambito. Anche qui gli stakeholder sono gli interlocutori;
	
	
	\item \textbf{Acceptable}, tutti gli stakeholder possono dire "yes! "; 
	designa lo stato di maturit\'a e dice che i requisiti
	sono uguali per tutti e tutti sono concordi;
	
	\item \textbf{Addressed}, ho la soluzione per i requisiti marcati come 
	acceptable, e sono capace di dimostrare che soddisfo tali requisiti. Questo 
	va dimostrato nella progettazione. "Sappiamo che abbiamo la soluzione";
	
	\item \textbf{Fullfilled}, "compiere". Ho un prodotto che fa esattamente 
	ci\'o che mi aspettavo.

	\item \textbf{Seeded}, si comincia a identificare di chi ho bisogno, non 
	come persone ma come competenze. Dobbiamo spersonalizzare, importa il 
	ruolo. 
	
	\item \textbf{Formed}, le persone c\'e le ho e posso lavorare; istituisco 
	dunque tecniche e strumenti per collaborare;
	
	\item \textbf{Collaborating}, le forme di collaborazione vanno studiate per 
	evitare perdite di tempo;
	
	\item \textbf{Performed}, una volta acquisite le tecniche siamo efficienti 
	ed efficaci;

	\end{itemize}	
	
	\item \textbf{Il project manager} si occupa della assegnazione delle 
	risorse e di stimare il problema in modo corretto, senza sovrastimarlo o 
	sottostimarlo, concentrandosi sui problemi per cui lo spazio dei rischi sia 
	conosciuto. Tiene conto del fatto che il personale non sar\'a mai a sua 
	completa disposizione, e si occupa di pianificare rispetto agli obiettivi, 
	eseguendo una pianificazione adattativa.


	\item I rischi latenti provocano problemi alla gestione di progetto:
	\begin{itemize}
		\item -->Variante nel personale, nella disponibilit\'a e nella 
		composizione del team. 
		Le persone possono	"sparire". A quel punto bisogna trovare un 
		rimpiazzo. Questa cosa va gestita, 
		devo fare un piano;
		
		\item -->Tecnologia, non sta mai ferma; due tecniche: chi usa solo 
		tecnologie consolidate (esempio uso di cobol nelle banche) perch\'e 
		cambiare produce rischio, e chi usa tecnologia 
		innovativa (per motivi di competizione) ma fortemente instabile. Questa 
		variabilit\'a \'e un grande rischio.
	\end{itemize}
		\item Alcuni rischi sono evitabili:
		\begin{itemize}
		\item Cambio di requisiti;
		
		\item Ritardo nelle specifiche.
		Gestione di rischi, vanno identificati all'inizio e in corso d'opera, 
		perch\'e possono variare nel tempo; 
		
		\item Identificazione, capire quali sono;
		
		\item Pianificazione, cosa posso fare per evitare i rischi;
		Un rischio si previene o lo si mitiga.	
		\end{itemize}
		


\item Il project manager vive costantemente sul risk management.
Il team necessita di ruoli, che identificano capacit\'a e compiti. Dentro 
un'organizzazione produttiva ci sono raggruppamenti di ruoli. Un ruolo \'e 
attivo su un progetto, la competenza di istanza si chiama funzione aziendale.

\item \textbf{Qualit\'a}, ci\'o che ci consente di puntare al miglioramento 
continuo di efficienza ed efficacia. Qualit\'a di prodotto \'e diverso da 
qualit\'a di processo. E' pi\'u importante la qualit\'a di 
processo;

\item \textbf{Sviluppo}, insieme delle competenze per la parte di attivit\'a 
tecnica e realizzativa del prodotto;

\item \textbf{Direzione}, fa s\'i che l'organizzazione possa stare in piedi;

\item \textbf{Amministrazione}, ("service manager") erogano/gestiscono 
l'infrastruttura che aiuta a fare il proprio lavoro (es. manutenzione e 
sicurezza). Se esiste una buona infrastruttura si 
lavora meglio.


	\item Competenze allo sviluppo:
	\begin{itemize}
		
	\item \textbf{Analista}, colui che serve per fare analisi dei requisiti; 
	aiuta l'avanzamento di maturit\'a dei requisiti; \'e
	molto importante e occorre avere competenze su molti fronti, sapere 
	ascoltare gli stakeholder, scrivere requisiti bounded e coherent, 
	ragionevoli, utili, realizzabili e verificabili. L'analista pensa al 
	problema, non alla soluzione;
	
	
	\item \textbf{Progettista}, pensa alla soluzione. Deve avere competenze 
	tecnologiche e tecniche. Ricevuto il problema tira fuori la migliore 
	soluzione possibile nel rispetto dei vincoli finanziari e temporali. 
	Competenze tecnologiche senza pregiudizi, ma in relazione alle necessit\'a. 
	Deve sapere di architettura, mettere insieme le parti della soluzione 
	(divide et impera). Rende il problema piccolo e dominabile. Le
	parti devono essere organizzate in modo da essere governabili. Queste parti 
	devono funzionare in un aggregato coerente. Un solo progettista in un team;
	
	
	\item \textbf{Programmatori}, il programmatore non inventa niente, deve 
	fare ci\'o che ha detto il progettista. Deve scrivere in modo non ambiguo. 
	Il passaggio di consegna tra progettista e programmatore deve essere
	chiaro. Permette il massimo parallelismo. 
	
	
	\item \textbf{Verificatori}, la validazione si fa sul prodotto finito. 
	Impiega almeno 1/3 del tempo. E' un ruolo attivo fi da subito. Serve per 
	creare l'ambiente giusto, a rendere puliti i 
	requisiti.
	
	
	\item \textbf{Responsabile}, ce ne sar\'a 1. Rappresenta il progetto presso 
	il fornitore e presso il committente. Svolge il ruolo di intermediario, che 
	dice "come siamo messi"; pianifica, gestisce risorse e rischi,
	coordina. Responsabilit\'a su relazioni esterne. Deve sapere ci\'o di cui 
	parla. 


	\item \textbf{Amministratore}, ruolo molto utile, prepara l'ambiente di 
	lavoro, strumento di collaborazione e controllo di avanzamento. Gestione 
	della documentazione di progetto. 
	Risoluzione di problemi legati alla gestione di processi. Assegnazione di 
	tickets.
	\end{itemize}

\end{itemize}

\section{Lezione 6 Gestione di progetto}
\begin{itemize}

\item \textbf{Gestione qualit\'a} la possiamo guardare da due punti di vista: 
qualit\'a sul prodotto e qualit\'a su ci\'o che si fa (qualit\'a di processo). 
E' una competenza strategica, una funzione aziendale e non un ruolo di progetto.
Serve come indicatore che stiamo lavorando bene". 


\item \textbf{Pianificazione di progetto} serve a sapere cosa dobbiamo fare 
nell'assicurarci che le nostre attivit\'a producano esiti che aiutino la nostra 
collaborazione; regolare l'avanzamento. 
Cominciamo ad identificare cosa "c\'e da fare", pianifichiamo le attivit\'a.

Chi lo pu\'o fare? 
Quanto mi costa? 

\item Le attivit\'a hanno un 
flusso, hanno un inizio ed un'uscita. Si 
distendono su un asse temporale.

\item Capite le attivit\'a mi devo chiedere quali ruoli assegnare. Attivit\'a 
che hanno dei vincoli sugli estremi ed eventualmente in mezzo. Devo capire 
quanto mi servir\'a per svolgere un'attivit\'a ad esempio: data,
tempo, persone, denaro.

\item Le ore vanno intese come ore di calendario: 
tempo/persona, quanto tempo uso tali persone sapendo che ciascuna percepisce un 
costo orario preciso. In base a questo posso capire quanto mi coster\'a un 
progetto.

\item \textbf{Strumenti per la pianificazione}:
\begin{itemize}

\item \textbf{Diagrammi di Gantt} strumento intellettuale semplice per mettere 
in relazione attivit\'a. 

Prima cosa da fare: 

\item determino le attivit\'a da svolgere su tutta la durata o su periodi 
pi\'u piccoli. L'orizzonte \'e dato in parte dal modello di ciclo di vita che 
adotto.

\item determinare le dipendenze, che consumano tempo e non producono. 

\item dire quanto tempo e persone mi serviranno per una certa attivita;


\item Una volta decise le risorse, alloco persone alle attivit\'a. Poi mettiamo 
a diagramma le cose e vediamo se tornano, altrimenti si torna indietro (ciclo). 

\item Il diagramma descrive, sull'asse orizzontale il tempo segmentato per
unit\'a di lavoro (es. 1gg, una settimana, un mese..). Una buona scelta \'e 
utilizzare come unit\'a di misura un giorno (8 ore lavorative); 

Sull'asse verticale, ci sono tutte le attivit\'a che devo svolgere e archi che 
rappresentano le transizioni tra una e l'altra.



\item \textbf{PERT} (Program Evaluation and Review Technique), serve a 
calcolare il rischio di calendario che consiste nell'avere un realistico 
margine tra un input e un output (slack). Se misuro lo slack e mi accorgo che 
\'e 0 o negativo ho una criticit\'a.. Non si pu\'o avere nemmeno
slack troppo grande(slack=differenza tra due date). Diagrammi specifici 
collocano gli slack e rappresentano il rischio. Il diagramma di Pert indica 
inoltre l'ordine delle attivit\'a e la loro durata e indica la prima data in 
cui posso iniziare un'attivit\'a che abbia tutti gli input necessari. 
Calcolando il cammino pi\'u lungo tra l'inizio e la fine  del progetto 
posso calcolare l'attivit\'a che finisce prima. Questo diagramma legge il Gantt 
rispetto allo slack e calcola i 
cammini critici.


\item \textbf{Work Breakdown Structure}, strumento per capire 
quali sono le attivit\'a utili/fondamentali tramite decomposizione (struttura 
di decomposizione del lavoro). L'attivit\'a viene scomposta e numerata con 
numerazione gerarchica. Ogni componente deve essere univocamente identificabile.
Un'attivit\'a decomposta permette di agevolare il parallelismo. Il WBS 
serve ad avere una percezione delle attivit\'a da svolgere.



\item \textbf{Allocazione delle risorse}: 
assegnare attivita a ruoli e poi ruoli a persone per far si che ogni persona 
sia impegnata per tutto il tempo disponibile. Diagramma di Gantt ribaltato 
sulle persone (al posto delle attivit\'a); l'intento e sapere che sto usando le 
persone in parallelo ed entro il limite dato. La decisione delle ore di 
investimento per la formazione personale e a discrezione di ciascuno. La 
componente informatica moderna e basata sull'apprendimento di tecnologie nuove. 


\item Fattori di influenza:
\begin{itemize}
	
	\item Dimensione del progetto: quante righe di codice mi servono; 
	
    \item Esperienza del dominio: quanta e l'esperienza di chi svolge il 
    progetto nei confronti del dominio a cui siamo davanti.
    
	\item Tecnologie adottate: influenza molto forte, e una scelta strategica 
	di amministrazione.
	
	\item Ambiente di sviluppo: procedure e tecniche con le quali collaboriamo, 
	dobbiamo collocarle a basso costo temporale.

 	\item Qualit\'a richiesta dai processi.
 	
\end{itemize}

\end{itemize}
\end{itemize}
\end{itemize}
\end{document}
